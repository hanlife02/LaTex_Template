% document.tex

% 手动设置标题、作者和日期的位置,并右对齐
\begin{flushright}
    \vspace*{7cm}  % 控制标题与页面顶部的距离
    \Huge\bfseries  % 增大字体并加粗
    \textcolor{blackk}{\textbf{标题内容}}\\[1cm]  % 标题加粗,设置为黑色
    \huge\bfseries  % 大号字体并加粗
    \href{https://hanlife02.com.cn/}{\textcolor{blackk}{Ethan}}\\[1cm]  % 超链接作者姓名,设置为黑色并加粗
    \huge\bfseries
    \textcolor{blackk}{han@hanlife02.com.cn}\\[1cm]
\end{flushright}

\begin{flushleft}
    \vspace*{7.5cm}
    \large\bfseries  % 增大并加粗字体
    \textcolor{blackk}{创建于:}
    \textcolor{red}{2024 年 9 月 16 日}\\[1cm]
    \large\bfseries 
    \textcolor{blackk}{更新于:}
    \textcolor{red}{\today}
\end{flushleft}

\newpage

% 目录
\tableofcontents
\newpage

% 第一章:正文内容示例
\section{正文示例章节}

这是一个简单的正文内容展示。你可以看到,普通文本使用的是默认的字号和对齐方式。你可以自由添加内容并调整格式。

\subsection{子章节标题}

这是一个子章节的内容展示。你可以在这里添加更多详细的文本内容,或者分段介绍不同的主题。

\subsection{列表示例}

这里是无序列表和有序列表的示例:

\paragraph{无序列表}
\begin{itemize}
    \item 项目 1
    \item 项目 2
    \item 项目 3
\end{itemize}

\paragraph{有序列表}
\begin{enumerate}
    \item 第一项
    \item 第二项
    \item 第三项
\end{enumerate}

\subsection{表格示例}

这是一个简单的表格展示:

\begin{table}[htbp]
\centering
\caption{示例表格}
\begin{tabular}{|c|c|c|}
\hline
\textbf{列1} & \textbf{列2} & \textbf{列3} \\ \hline
数据 1      & 数据 2      & 数据 3      \\ \hline
数据 4      & 数据 5      & 数据 6      \\ \hline
\end{tabular}
\end{table}

\newpage

\section{代码示例}

这是一个代码块的展示,支持语法高亮:

\begin{lstlisting}[language=Python, caption=Python 代码示例]
def hello_world():
    print("Hello, World!")
\end{lstlisting}

\newpage

\section{插图示例}

\begin{figure}[htbp]
\centering
\includegraphics[width=0.8\textwidth]{example-image-a}
\caption{示例图片}
\end{figure}

\newpage

\section{彩色信息框示例}

\begin{info}[重要信息]
    这是一条带有标题的彩色信息框,用于展示重要的内容提示。
\end{info}